\documentclass[sigconf]{acmart}

\usepackage{multirow,graphicx}
\usepackage{hyperref}


\usepackage[]{natbib}
\bibliographystyle{plainnat}

\makeatletter
\@ifpackageloaded{subfig}{}{\usepackage{subfig}}
\@ifpackageloaded{caption}{}{\usepackage{caption}}
\captionsetup[subfloat]{margin=0.5em}
\AtBeginDocument{%
\renewcommand*\figurename{Figure}
\renewcommand*\tablename{Table}
}
\AtBeginDocument{%
\renewcommand*\listfigurename{List of Figures}
\renewcommand*\listtablename{List of Tables}
}
\@ifpackageloaded{float}{}{\usepackage{float}}
\floatstyle{ruled}
\@ifundefined{c@chapter}{\newfloat{codelisting}{h}{lop}}{\newfloat{codelisting}{h}{lop}[chapter]}
\floatname{codelisting}{Listing}
\newcommand*\listoflistings{\listof{codelisting}{List of Listings}}
\makeatother

\begin{document}

\title{The impact of opt-in gamification on \\ students' grades in a
software design course}
\author{Kiko Fernandez-Reyes}
  \affiliation{\institution{Uppsala University}}
  \email{kiko.fernandez@it.uu.se}
\author{Dave Clarke}
  \affiliation{\institution{Uppsala University}}
  \email{dave.clarke@it.uu.se}
\author{Janina Hornbach}
  \affiliation{\institution{Uppsala University}}
  \email{janina.hornbach@fek.uu.se}
\date{}

\begin{abstract}
An achievement-driven methodology strives to give students more control
of their learning with enough flexibility to engage them in deeper
learning.

We observed in the course Advanced Software Design, which uses the
achievement-driven methodology, that students fail to get high grades,
which may hamper deeper learning. To motivate students to pursue and get
higher grades we added gamification elements to the course.

To measure the success of our gamification implementation, students
filled out a questionaire rating the enjoyment and motivation produced
by the game. We built a statistical regression model where enjoyment and
motivation explain 55\% of the variation in grades. However, only the
relationship between motivation and grade is significant, which implies
that notivation drives the overall effect of the model. The results
suggest that the more the students were motivated by the game, the
higher their grades on the course (and vice versa). This implies that if
gamification indeed motivates students, then it makes them go beyond
what is expected.
\end{abstract}

\copyrightyear{2018}
\acmYear{2018}
\setcopyright{acmlicensed}
\acmConference[MODELS '18 Companion]{ACM/IEEE 21th International Conference on Model Driven Engineering Languages and Systems}{October 14--19, 2018}{Copenhagen, Denmark}
\acmBooktitle{ACM/IEEE 21th International Conference on Model Driven Engineering Languages and Systems (MODELS '18 Companion), October 14--19, 2018, Copenhagen, Denmark}
\acmPrice{15.00}
\acmDOI{10.1145/3270112.3270118}
\acmISBN{978-1-4503-5965-8/18/10}

\begin{CCSXML}
<ccs2012>
<concept>
<concept_id>10010405.10010489</concept_id>
<concept_desc>Applied computing~Education</concept_desc>
<concept_significance>500</concept_significance>
</concept>
<concept>
<concept_id>10003456.10003457.10003527.10003531.10003751</concept_id>
<concept_desc>Social and professional topics~Software engineering education</concept_desc>
<concept_significance>300</concept_significance>
</concept>
<concept>
<concept_id>10011007.10011006.10011060.10011061</concept_id>
<concept_desc>Software and its engineering~Unified Modeling Language (UML)</concept_desc>
<concept_significance>300</concept_significance>
</concept>
</ccs2012>
\end{CCSXML}

\ccsdesc[500]{Applied computing~Education}
\ccsdesc[300]{Social and professional topics~Software engineering education}
\ccsdesc[300]{Software and its engineering~Unified Modeling Language (UML)}


\keywords{gamification, education, software design, UML}

\maketitle




\hypertarget{introduction}{%
\section{Introduction}\label{introduction}}

The course \textit{Advanced Software Design} is built around the
achievement-driven learning methodology \citep{wrigstad2017mastery}
(more details in Section~\ref{sec:theory}). This methodology gives
students more control of their learning with enough flexibility to
engage them in deeper learning \citep{biggsTang2011}. Yet, year after
year, although the course \textit{Advanced Software Design} has
increased in the number of students, we observed that students do not
pursue higher grades. In this course, a higher grade implies the
possibility of deeper learning, as students need to justify the design
choices of the software-under-construction and they cannot just memorise
solutions (Section \ref{sec:theory} explains more details).

To motivate students to pursue higher grades and deeper learning, we
designed a new opt-in game on top of the achievement-driven learning
methodology, adding gaming elements such as leaderboard, cards and
points. To answer whether the enjoyment and motivation provided by the
game has an effect on grades we built a regression model that explains
55\% of the variation in grades.

The paper makes the following contributions:

\begin{itemize}
\item Gamification of the achievement-driven learning method
\item Evidence that students motivated by the game get higher grades
\item Implementation experience.
\end{itemize}

\hypertarget{sec:theory}{%
\section{Theory}\label{sec:theory}}

\hypertarget{sec:achievement-driven}{%
\subsection{Achievement-driven learning
methodology}\label{sec:achievement-driven}}

The learning outcomes specified to help students succeed with their
courses \citep{biggs1996} are often necessarily at a high level, for
instance to allow course content to change more fluidly, and thus
arguably do not help students as much as they should. Five to ten goals
might be specified for a course, but this does not tell a student what
needs to be done to earn a specific grade. Indeed, often the learning
outcomes can only be fully understood by someone who has completed the
course.

To make these learning outcomes more achievable, comprehensible and
accessible, Tobias Wrigstad and Elias Castegren
\citep{wrigstad2017mastery} developed a new assessment technique, called
\textit{achievement-driven learning}, based on a more fine-grained
collection of achievements, which taken together encompass the course's
learning outcomes, but are more achievable when taken individually. The
list of achievements makes more explicit what a student needs to master
in order to pass the course, but also to achieve higher grades. In a
sense, achievement-driven learning refines constructive alignment to the
micro level and provides a means for linking learning outcomes with
assessment activities---it aims to hit a sweet spot in the constructive
alignment design space.

In achievement-driven learning, students are presented with a number of
achievements (21 in Advanced Software Design) that need to be
satisfactorily demonstrated to achieve each grade. Achievements are
divided into 3 groups (3, 4,
5)\footnote{In many courses at Uppsala University, grades are given from a 4-point scale, U (fail),
3 (pass), 4 (pass with distinction), and 5 (pass with excellence).}, and
students need to demonstrate all 3s to get a 3, all 3s and 4s to get a 4
and all achievements to get a 5. Demonstration of achievements involves
a dialogue with a teaching assistant, which creates a reason to interact
with knowledgable assistants beyond the usual correction of assignments
and troubleshooting, and feedback comes naturally. Feedback, of course,
enhances learning quality \citep{hattie2012}.

Achievement-driven learning adopts aspects of mastery learning
\citep{bloom1974}. Students are required to master all achievements
specified for each given grade, and achievements must be retried until
the teaching assistant is satisfied. This gives students freedom to
fail, but also ensures that students passing the course do actually
satisfy the learning outcomes---in contrast, grading schemes in which
students only need to obtain a certain number of points do not guarantee
coverage of the learning outcomes.

Typically, there are far too many achievements to demonstrate all
individually during the time available (certainly, to obtain higher
grades). To counter this, achievement-driven learning encourages
students to see the relationship between multiple achievements, and to
group and demonstrate them coherently together. Not only does this
reduce their workload, it forces students to search for the connections
between topics. Achievements can often be grouped vertically: level 5
achievements on a certain topic often encompass level 3 and level 4
achievements on the same topic---the quality of learning being
demonstrated is the key difference. Achievements can also be grouped
horizontally: a single presentation can be used to demonstrate
achievements addressing different topics, assuming that students
establish the relationship between the topics within their presentation.
In a sense, achievement-driven learning forces students to solve a
puzzle in order to optimise their time---similar to ideas underlying
gamification \citep{leeHammer2011}---and as a result they are compelled
to explore the connections between concepts. The idea of vertical
combinations of achievements is compatible with the requirement of the
Bologna Process that greater \textit{quality} of work \textbf{not}
greater \textit{quantity} earns higher grades
\citep{bolognaDeclaration}.

Achievement-driven learning has a lot in common with self-regulatory
learning, which sees successful learners as able to plan, set goals,
organise, self-monitor, and self-evaluate \citep{zimmerman1986}, then
the teaching assistant can stop the demonstration. Students are not only
forced to plan, but they need to evaluate whether their plan will be
good enough---although having a demonstration `failed' does not directly
affect a student's grade, only a finite number of opportunities to
demonstrate are available. In general students need to reflect on their
learning as a whole, and cannot simply hope to get by without engaging
with the material. Students are forced to determine their own learning
trajectory. Indeed, students are free to choose the most appropriate
form of examination, avoiding examination formats that do not match the
task at hand (e.g., a written exam for a software design process).
Achievement-driven learning puts more responsibility on students for
their own education, while providing more freedom in how they embrace
the subject. To clarify for the student what she actually can do,
thereby building both self-awareness and self-confidence, more so that
an standard collection of learning outcomes, and flexibility motivates
students to engage in deeper learning \citep{biggsTang2011}.

Achievement-driven learning is similar to Clark's approach based on
Student Observable Behaviours (SOBs) \citep{clark2013}. SOBs are similar
to achievements and software exists to support both schemes. A major
difference is that achievement-driven learning compels students to
combine achievements and understand the relationship between topics.

The achievement list is presented to students at the start of the course
and it is made clear which ones need to be completed in order to get
each grade (3, 4, 5). In this respect an achievement list resembles a
grading criteria, and literature on the impact of good grading criteria
on learning \citep{rustPrice2003} is arguably applicable to
achievement-driven learning. A key difference is the combination of
achievement that students must explore to construct their own
demonstrations.

\hypertarget{gamification}{%
\subsection{Gamification}\label{gamification}}

Gamification refers to the use of game elements in non-gaming
environments \citep{deterding2011game}. Gamification in education is not
a new idea
\citep{huang2013gamification, dicheva2015gamification, gamification-leaderboard-benefits}
and courses that use gamification elements do not put the emphasis on
the game, rather it uses the game as a motivational and driving factor.
Thus, the most important benefit is its potential to increase students'
motivation and engagement
\citep{SEABORN201514, gamification-leaderboard-benefits}. There are two
types of gamification: reward-based and meaningful \citep{oldwine}.
Reward-based uses badges, leaderboards and achievements -- external
elements -- as a way of measuring progress; meaningful gamification
strategies try to find a connection with the participant -- intrinsic
motivation. From the psychological point of view, these match with
extrinsic and intrinsic motivational factors
\citep[\citet{psychologyMotivation},\citet{RYAN200054}]{intrinsic-book}.
Motivation is \textit{extrinsic} if an individual is motivated by
external factors, e.g.~recognition after completion of a task and
\textit{intrinsic} if an individual experiences joy when performing a
task . The use of gamification in the learning process can trigger both
extrinsic and intrinsic motivation, which could induce deep learning and
higher grades.

Therefore we derive the following hypothesis:

\begin{description}
\item[H] Motivation and enjoyment from gamification increases grades.
\end{description}

\hypertarget{background}{%
\section{Background}\label{background}}

Advanced Software Design is a masters-level course taught in the
Department of Information Technology at Uppsala University. The course
covers topics such as object-oriented analysis and design, domain
modelling, software architecture, class and object modelling,
behavioural modelling, design patterns, GRASP principles, design
evaluation, and design improvement/refactoring. The course does not
involve any programming, and instead uses UML, text, and oral
presentation as the means for recording and communicating designs.
Students taking the course are expected to have a solid background in
programming, in particular in object-oriented programming. The course
offers 8--10 interactive lectures describing the material and a
course-long project. Students form teams on their own (of 4 members and
they are encouraged to diversify the team's skill set) and work together
on the achievements. Teams get a fixed teacher assistant (TA) for the
duration of the course. Each team has a weekly, thirty-minute-long
meeting and feedback sessions with a TA. During the meetings team
members are evaluated individually, i.e.~some team members may pass an
achievement while others will have to try again in the next meeting. The
assessment scheme for the project is based on achievement-driven
learning \citep{wrigstad2017mastery}, as described in Section
Section~\ref{sec:achievement-driven}. There is a total of 21
achievements\footnote{Link to the the overview, achievement goals and requirements: \url{https://goo.gl/CNLLLg}}:
11 achievements of level 3, 9 achievements of level 4 and a single
individual achievement of level 5. There is no exam.

\hypertarget{sec:methodology}{%
\section{Methodology}\label{sec:methodology}}

\begin{figure}[t]
\begin{tabular}{|c|c|}
\hline
Achievement & Points \\
Grade & \\
\hline
3 & 50 \\
\hline
4 & 100 \\
\hline
\end{tabular}
\caption{\label{fig:point-system}Number of points of each achievement. Students get \textit{floor(points/100)} cards per meeting.}
\end{figure}

\begin{figure}[t]
\begin{tabular}{|r|l|l|}
\hline
Category & Name & Overview \\
\hline
& Thief & Steal 20 points from \\
&& leading team \\ \cline{2-3}
Attacks & Master Thief & Steal 20 points from \\
&& two teams \\ \cline{2-3}
& Destroyer & Reduce 50 points of \\
&& chosen team \\ \cline{2-3}
& Longinus Spear & Impale up-to 3 teams, steal \\
&&20 points from each team \\ \cline{2-3}
\hline
& Safe pockets & Protection against Thieves  \\ \cline{2-3}
Shields & Safe box & Protection against \\
&& (Master) Thieves \\ \cline{2-3}
& Self preservation & Protection against Destroyer  \\ \cline{2-3}
\hline
Others & Death note & Forbids chosen team from \\
&& making a move\\ \cline{2-3}
& Sell out & System buys card for 30 points \\ \cline{2-3}
\hline
\end{tabular}
\caption{\label{fig:available-cards}Cards available in the game.}
\end{figure}

\hypertarget{sec:the-game}{%
\subsection{The Game}\label{sec:the-game}}

We developed a game of top of the achievement-driven learning
methodology. This game uses points, leaderboards and cards and is played
at a meta-level, i.e. the game is not related to the software design
process but rather it uses the achievements passed by teams during the
meetings as its internal structure, e.g.~to give cards to a team so that
they can use them to perform some special action in the game such as
stealing points from other team.
\textit{Students earn points by passing achievements and by playing the game, but
points won or lost in the game do not influence the number of achievements passed or the final grade.}

Students form teams and teams are place randomly in mini-competitions
(made up of 4 teams) that are course-wide; the team with most points
wins. Each achievement has a fixed number of points, linked to its level
(Figure \ref{fig:point-system}): students who pass achievements of level
3 get 50 points, students who pass achievements of level 4 get 100
points. Teams demonstrate their knowledge to the TA and gain the total
sum of the points in the achievements that they successfully pass each
week. The more points a team gets, the more cards a team can draw; teams
draw \textit{floor(number of points/100)} cards per meeting (cards are
randomly drawn from an online number generator). Cards allow teams to
attack to each other (subtracting points), steal points
(subtract-and-gain points), block teams from using cards and even raise
their own score (Figures \ref{fig:available-cards} and
\ref{fig:ncards}). Each card has a priority number, which is randomly
generated when students get a card, and this number establishes the
order in which the cards are applied, within each round of the
competition. Cards with higher priority number will be played before
cards with a lower number. This system removes the advantage that some
teams may get by scheduling meetings early in the week since the
priority number is randomly generated.

Teams may decide to keep cards and use them all in the last meeting of
the game or they may decide to play them as they come. Ultimately, the
strategy is in their own hands.

Each week the TAs have meetings with the teams, who prove their
knowledge presenting achievements and the teams get the corresponding
points and cards. Teams tell TAs which cards they would like to use and
the TAs collect this information (cards and their priorities). Based on
this information, the TAs make up a story timeline of what happens when
they applied the cards (based on their priorities), always trying to be
funny and unpredictable as well as to encourage students when they put
effort and performed well.

For example, the team \textit{Yin Yang} passed at least 2 achievements
of level 3 in week 46 (granting them a card) and they got a
\textit{Thief} card with a high priority number. They decided to use it
that week, which led to the following comment in the story timeline:

\begin{quote}
Yin Yang are sneaky and steal from (team) Number One, 20 points (using a Thief card).
Number One is still on the lead, but not by far. This is going to be a tough competition.
\end{quote}

An example of encouragement words to keep them motivated:

\begin{quote}
Team Number One got 100 points again.
They seem to have found the sweet spot to tick off achievements!
\end{quote}

The story timeline shows the classification before and after the
application of cards.

\hypertarget{gaming-design}{%
\subsection{Gaming design}\label{gaming-design}}

The game is optional and does not link the game to the final grade
(avoids creation of parallel assessment routes \citep{glover2013play}).
If the students decide to not participate, the course takes place as in
previous years. From the beginning, we told students that they have
nothing to lose and they should try the game.

The main idea was to foster competition, motivation, engagement and
higher grades with the introduction of the gaming elements (points, a
leaderboard and cards). As mentioned before, students form teams placed
randomly in mini-competitions; the use of mini-competitions serves to
keep teams always close to each other, preventing a team being put in a
far-from-the-top position, which can be discouraging
\citep{gamification-brain-trust}.

To make the game appealing, we added the card system and we release new
cards each week, to prevent stagnation. The introduction of cards and
leaderboard for the mini-competition introduces extrinsic motivation
\citep{zichermann2011gamification}.

To counteract the fact that students have meetings at different times
during the week, the game is not played in real time, but cards are
played at the end of each week based on a random priority assigned to
the cards, as explained in Section Section~\ref{sec:the-game}.

Finally, we created a story timeline, so that students know why they get
different points from the ones collected during the meetings (due to
attacks, shields, etc).

In terms of platform, we used Google
Spreadsheets\footnote{https://goo.gl/iZLS5p} with three tabs:

\begin{description}
\item[Leaderboard:] contains the leaderboard and the rules
\item[Timeline:] contains the timeline for each mini-competition
\item[Cards:] contains the number of cards, which ones have been just released and clear explanation
\end{description}

The main reason for using Google Spreadsheets was that it is simple to
use, platform agnostic (no need to commit to any platform) and let us
connect Google Analytics to gain further insights.

\hypertarget{evaluation-questionnaire-and-variable-measurement}{%
\section{Evaluation: Questionnaire and variable
measurement}\label{evaluation-questionnaire-and-variable-measurement}}

As part of evaluating the gamification methodology students were asked
to answer two questions about their experience with the game. First, we
asked to which extent they enjoyed the game, which became our
independent variable ``Enjoy''. Students answered on a Likert scale from
1 (``Not at all'') to 5 (``Loved it''). Second, students reported on a
Likert scale from 1 (``I didn't bother'') to 5 (``I tried to get as many
as possible'') to which degree the game competition motivated them to
get more cards, which became our independent variable ``Motivation''.

Our dependent variable ``Grade'' was computed from 2 to 5, 2 being
suspended, 3 being pass, 4 being pass with distinction, and 5 being pass
with excellence. The boxes ``Enjoy'', ``Grade'' and ``Motivation''
(Fig.~\ref{fig:matrix}) show the distribution and frequency of responses
for our 3 variables.

Questionnaire responses were anonymous within teams, meaning that we
could not identify individual responses or names within teams, except
for 3 individuals who got different grades than their team members and
were personally asked whether they wanted to share their answer for data
matching purposes.

\hypertarget{results-and-discussion}{%
\section{Results and Discussion}\label{results-and-discussion}}

\begin{table*}[t]
\centering
\begin{tabular}{|r|r|r|r|r|r|r|r|r|r|r|r|r|}
\hline
 & vars & n & mean & sd & median & trimmed & min & max & range & skew & kurtosis & se \\
\hline
Enjoyment & 1 & 71 & 2.79 & 1.57 & 3 & 2.74 & 1 & 5 & 4 & 0.10 & -1.60 & 0.19 \\
\hline
Motivation & 2 & 71 & 2.65 & 1.64 & 2 & 2.56 & 1 & 5 & 4 & 0.22 & -1.67 & 0.19 \\
\hline
Grade & 3 & 71 & 3.65 & 0.81 & 3 & 3.60 & 2 & 5 & 3 & 0.39 & -0.91 & 0.10 \\
\hline
\end{tabular}
\caption{\label{tbl:statistics}Descriptive statistics. \emph{sd} stands for \emph{standard deviation}
and \emph{se} for \emph{standard error}}
\end{table*}

Statistical analyses were performed using R version 3.5.0.

71 students answered the questionnaire providing 71 observations without
missing data. Table \ref{tbl:statistics} shows descriptive statistics of
our 3 variables, including means, standards deviations, skewness and
kurtosis scores. The Pearson correlation matrix in Table
\ref{tbl:pearson} shows that both ``Motivation'' and ``Enjoyment'' are
positively correlated with our dependent variable ``Grades''. The
correlation between the two independent variables is below \(0.90\),
meaning that we do not seem to have issues with multicollinearity. The
scatterplot matrix (Figure \ref{fig:matrix}) confirms furthermore that
the relationships between our 3 variables are linear. For example, the
upper right box shows a linear correlation between ``Motivation'' on the
x-axis and ``Enjoyment'' on the y-axis.

\begin{table}[t]
\begin{tabular}{|r|l|l|l|}
\hline
 & Enjoyment & Motivation & Grade \\
 \hline
 Enjoyment & 1.0000000 & 0.8659292 & 0.6927165\\
 \hline
 Motivation & 0.8659292 & 1.0000000 & 0.7313952 \\
 \hline
 Grade & 0.6927165  & 0.7313952 & 1.0000000 \\
 \hline
\end{tabular}
\caption{\label{tbl:pearson}Pearson correlation}
\end{table}

\begin{table}
\begin{tabular}{|r|l|l|l|l|}
\hline
& Estimate & Std. Error & t value & Pr(>|t|)  \\
 \hline
(Intercept)  & 2.61479 &   0.13519  & 19.342 & < 2e-16 *** \\
 \hline
Enjoyment &        0.12307 &    0.08442 &   1.458 &  0.14949     \\
 \hline
Motivation &   0.26054 &    0.08067 &   3.230 &  0.00191 **  \\
 \hline
 \multicolumn{5}{|l|}{ Signif. codes:  0 '***' 0.001 '**' 0.01 '*' 0.05 '.' 0.1 ' ' 1} \\
 \hline
 \multicolumn{5}{|l|}{Residual standard error: 0.5535 on 68 degrees of freedom} \\
 \multicolumn{5}{|l|}{Multiple R-squared:  0.549,   Adjusted R-squared:  0.5358 } \\
 \multicolumn{5}{|l|}{ F-statistic: 41.39 on 2 and 68 DF,  p-value: 1.741e-12}\\
 \hline
\end{tabular}
\caption{\label{tbl:regression}Regression analysis}
\end{table}

\begin{table}[t]
\begin{tabular}{|r|l|l|}
\hline
 & W & p-value \\
 \hline
 Residuals & 0.89516 & 2.223e-05 \\
 \hline
\end{tabular}
\caption{\label{tbl:shapiro}Shapiro-Wilk normality test}
\end{table}

\begin{figure}[t]
\includegraphics[width=0.5\textwidth]{data/scatterplots/RScatterplotMatrix2.png}
\caption{\label{fig:matrix}Scatterplot matrix}
\end{figure}

\begin{figure}
\includegraphics[trim=0 0 0 0.7cm,clip,width=0.5\textwidth]{data/Rplot/Rplotqq.png}
\caption{\label{fig:qq} Normal QQ plot}
\end{figure}

\begin{figure}
\includegraphics[width=0.5\textwidth]{data/Rplot/RplotHistResidual.png}
\caption{\label{fig:res-hist} Residual histogram}
\end{figure}

Results of our linear regression analysis (Table \ref{tbl:regression})
show that ``Enjoyment'' and ``Motivation'' explain 55\% of the variance
in students' grades. The critical cutoff value of the F-distribution at
a 5\% significance level is 3.13. Since our F-value of 41.39 is larger
than 3.13, we can conclude that our model fits the data well. However,
only the relationship between ``Motivation'' and ``Grade'' is
significant (t-value 3.20, p\textless0.01), which implies that
``Motivation'' drives the overall effect of the model. The relationship
between ``Enjoyment'' and ``Grade'' is not significant. The results
suggest that the more the students were motivated by the game, the
higher their grades on the course (and vice versa). Consequently, our
hypothesis is partially confirmed.

Tests for normality indicate that our residuals deviate from a normal
distribution (Shapiro-Wilk test: p-value \textless{} 0.05, Table
\ref{tbl:shapiro} and residual QQ plot, Figure \ref{fig:qq}). However,
the residual histogram shows that the distribution looks symmetric and
bell-shaped (Figure \ref{fig:res-hist}). The impact of non-normality
depends on both distribution and sample size
\citep{hair2010multivariate}. According to \citep{hair2010multivariate},
small sample sizes below 50 observations might be problematic, while
larger sample sizes increasingly cancel out violations against
normality. Given that our sample size is 71, we believe that results
from our regressions are still accurate despite non-normality of the
residuals.

\hypertarget{sec:implementation}{%
\section{Implementation experience}\label{sec:implementation}}

\begin{figure}[t]
\begin{tabular}{|r|l|l|}
\hline
Category & Name & Overview \\
\hline
& Mini-turn & Get 15 min extra feedback  \\
& (retired) & \\ \cline{2-3}
Wisdom & Extra turn & Get 15 min extra to \\
& (not available) & tick off achievements \\ \cline{2-3}
& E-Deadline & Get 24h to re-submit  \\
& (not available) & after feedback from TA \\ \cline{2-3}
\hline
Others & Eavesdrop & Show hand (cards) of all teams \\
\hline
\end{tabular}
\caption{\label{fig:ncards}Problematic cards.
Cards retired from the game are marked as \emph{retired}.
Cards that were in the released schedule but
never went live are marked as \emph{not available}.}
\end{figure}

In this section we discuss our experience when using gamification
elements in the course Advanced Software Design, focusing on:

\begin{itemize}
\item Time to create a game
\item Designing a game for maintainability
\item Educational laws and games
\item Gamers need help
\item Implementation.
\end{itemize}

\hypertarget{time-to-create-a-game}{%
\subsubsection*{Time to create a game}\label{time-to-create-a-game}}
\addcontentsline{toc}{subsubsection}{Time to create a game}

Creating a fair game is no easy task and requires \emph{plenty} of time.
Fairness means that the best team should win. You should plan with
enough time to come up and design your gaming elements. In our setting,
gaming elements are cards and points, whether to play in real time or
using a priority system and its consequences, among other things. Do not
underestimate the time that it takes to write unambiguous game rules and
ask for feedback on the wording. From our experience, we were a diverse
group of TAs (one Spanish, one Chinese and one Vietnamese) and the game
rules were more understandable after a common discussion.

\hypertarget{designing-a-game-for-maintainability}{%
\subsubsection*{Designing a game for
maintainability}\label{designing-a-game-for-maintainability}}
\addcontentsline{toc}{subsubsection}{Designing a game for
maintainability}

We wanted to create an engaging game that, together with the course,
could be maintained by two TAs working full time (120 h for the duration
of the course, 10 weeks) and one TA working a maximum of 60 h. To do
this, teams should \emph{not} be able to get too many cards per week or
the game would not be maintainable. The non-maintainability comes from
playing actions (using cards), which need a story and to keep track of
the order in which each action is taken. For this reason, we tested
playing a few hands and we settled on
\textit{floor(number of points/100)} cards per meeting (Figure
\ref{fig:point-system}). Writing the story timeline and keeping the
scores up-to-date per action takes at least two hours per
mini-competition and we had 5 mini-competitions. Each TA was responsible
for writing their teams story timeline. Overall, our feeling is that the
gamification of the achievement-driven methodology is very TA intensive,
although manageable if well planned.

\begin{figure}[t]
\includegraphics[width=0.4\textwidth]{data/histograms/EnjoyMotivationHistogramGrey.png}
\caption{\label{fig:histogram-enjoy-motiv}Histogram showing students enjoyment and motivation by the game.
Numbers from 1--5 represent the Likert degree (1 lowest, 5 maximum).}
\end{figure}

\begin{figure}[t]
\includegraphics[width=0.4\textwidth]{data/histograms/RplotGradesBoxed.png}
\caption{\label{fig:game-grades}Histogram showing students grades}
\end{figure}

\begin{figure}[t]
\includegraphics[width=0.4\textwidth]{data/histograms/EnjoyMotivationHistogram_TAGrey.png}
\caption{\label{fig:game-enjoy-support-ta}Histogram showing students enjoyment and motivation by the game,
with support from TA.
Numbers from 1--5 indicate the degree (1 lowest, 5 maximum)}
\end{figure}

\begin{figure}[t]
\includegraphics[width=0.4\textwidth]{data/histograms/RplotHistogramTA.png}
\caption{\label{fig:game-grades-support-ta}Histogram showing students grades
when the TA provides support during the game}
\end{figure}

\subsubsection*{Educational laws and games}

Regarding the cards and their utility, we had more cards than the ones
that were available (Figure \ref{fig:available-cards} and
\ref{fig:ncards}). Wisdom cards were particularly problematic (Figure
\ref{fig:ncards}): the card \emph{Mini-turn} had to be retired after the
second week due to its potential unfair treatment to students, i.e. some
students could receive more feedback than others. (This is not allowed
under the Swedish Educational Law.) Based on the same principle, cards
\emph{Extra turn} and \emph{E-deadline} were going to be released later
on but never saw the day of light. One needs to be careful when
designing and implementing a game and know the educational laws of the
country in which the course takes place.

\hypertarget{gamers-need-help}{%
\subsubsection*{Gamers need help}\label{gamers-need-help}}
\addcontentsline{toc}{subsubsection}{Gamers need help}

One thing to improve was to be consistent in how we encourage and
support students to play the game. TAs were instructed to encourage
students to play the game. One TA added 10 extra minutes per group
meeting to provide support to students by answering questions regarding
the game, cards and leaderboard score. The result was that most of his
students were engaged, highly motivated by the game and they got better
grades (overall result in Figure \ref{fig:histogram-enjoy-motiv} and the
overall grades in Figure \ref{fig:game-grades}; subset of students with
support from the TA in Figure \ref{fig:game-enjoy-support-ta} and their
grades in Figure \ref{fig:game-grades-support-ta}).

The other TAs did not add these 10 extra minutes to their meetings and,
when students asked questions regarding the rules and cards, they
referred the students to the online documentation. These students did
not participate in the game that much and some teams thought that the
achievement-driven learning methodology was already too complex to add a
game on top of.

\hypertarget{implementation}{%
\subsubsection*{Implementation}\label{implementation}}
\addcontentsline{toc}{subsubsection}{Implementation}

During the game competition we observed that fostering competition makes
students go beyond what is expected to lead the classification;
something that was also discovered in a previous study
\citep{gamification-leaderboard-benefits}.

\hypertarget{summary}{%
\subsubsection*{Summary}\label{summary}}
\addcontentsline{toc}{subsubsection}{Summary}

Creating a game takes time and we recommend not to take this lightly
\citep{review-gamification-framework}. In the integration of your game
and course, always think about the effort that the gamification elements
entail, i.e.~adding game elements that are easy to integrate with your
course and that you can maintain. The game should be easy to understand,
or students will not make the effort to play. Finally, make sure that
the game satisfies the educational laws of your country (check them!).

\hypertarget{limitations-and-threats-to-validity}{%
\section{Limitations and threats to
validity}\label{limitations-and-threats-to-validity}}

A limitation of the study is related to the use of cross-sectional data,
as we cannot claim causality of the effects of gamification. However,
given that students filled out questionnaires before they knew their
grades helps us to infer that motivation most likely affected grades,
and not that better grades made students more motivated. Regarding
construct validity, another limitation might have been the use of
one-item measures for our independent variables.

Another threat to validity refers to the achievement-driven methodology
applied to a software design course: TAs were trained to evaluate
software designs and welcomed to talk to the main lecturer in case of
doubts regarding designs. However, whether team members pass or fail an
achievement is completely subjective to the TA's opinion. Regarding the
encouragement and support from the TAs (to students) to play the game,
the TAs come from different cultures and the way they provide
encouragement and support may be differently shown. This could affect
the validity of the subsection \textit{Gamers need help}
(Section~\ref{sec:implementation}).

\hypertarget{related-work}{%
\section{Related work}\label{related-work}}

In our case, the steps to the gamification of the course closely follow
the suggested guidelines by De Paz \citep{gamification-thesis}. Points
of departure were related to issues outside of our control, such as,
gathering team members and knowing your players. In the former, the TAs
have the knowledge but we lack experience (consistency between TAs,
Section~\ref{sec:implementation}). In the latter, the course takes
Swedes and international students and students come from different
backgrounds, which made difficult finding a unifying game that could
satisfy them all.

The achievement-driven learning \citep{wrigstad2017mastery} uses
gamification to force students to solve a puzzle in order to optimise
their time. This work extends achievement-driven learning, introducing
explicit gaming elements to foster competition, engagement and the
possibility of deeper learning.

Gamification of computer science courses that add gaming elements often
use points, leaderboards and badges and report on higher engagement from
students
\citep{eng-engineering-gamification, freitas-twice, todor, Villagrasa, ODonovan}.
Instead, our work uses similar gaming elements although we measure
whether gamification elements induce higher grades.

The gamification of a course does not always succeed, even when one adds
leaderboards, points and other gaming elements, as noted by K. Berkling
et al \citep{Berkling}. In this paper, we report mixed feelings from
students (Figure \ref{fig:histogram-enjoy-motiv}), where we found two
extremes, students who loved the game and students who didn't like it.
However, we found that adding 10 extra minutes to discuss the game
mechanics can have a huge impact on the motivation, enjoyment of the
game and students' grades (Figures \ref{fig:game-enjoy-support-ta} and
\ref{fig:game-grades-support-ta}, Section~\ref{sec:implementation}).

We report our implementation experience, something that de Sousa Borges
argues is often forgotten
\citep{deSousaBorges:2014:SMG:2554850.2554956}. In this regard, our
experience coincides with O'Donovan's work \citep{ODonovan}, that is,
the creation of a game takes time. O'Donovan's report that the
gamification of a course can incur in high (monetary) costs, as they
hired a programmer and designer to create a game. In our setting, we
used a platform-free software, i.e.~the free Google Excel Sheet.

\hypertarget{conclusion}{%
\section{Conclusion}\label{conclusion}}

We have added gaming elements to the \textit{Advanced Software Design}
course, based on the achievement-driven learning methodology
\citep{wrigstad2017mastery}, to motivate students to get higher grades.
The gaming elements provide enjoyment and motivational factors. We have
built a regression model where enjoyment and motivation explain 55\% of
the variation in grades, whereby motivation drives the overall effect of
the model, meaning that students who were motivated by the game also got
higher grades. The link between enjoyment and grades, however, was
insignificant. A future research direction could be to further explore
the motivational factors that drive higher grades. Are students
motivated intrinsically or extrinsically by opt-in gamification
elements? What is it that drives motivation when it comes to
gamification? If enjoyment, which was insignificant in our model, is
indeed an intrinsic motivational factor, as suggested by
\citep{RYAN200054}, our results suggest that intrinsic motivational
factors may be less likely to influence students' grades after all. Our
results provide a hint that extrinsic motivational factors in form of
achievements may be more important in students' motivation. However,
future research is needed to corroborate those suggestions.

Finally, we report on the implementation of the game and remark that
adding 10 extra minutes per meeting can potentially have a positive
effect on students enjoyment, motivation of the game and thus, higher
grades.


      
    \bibliography{biblio.bib}

  

\end{document}
